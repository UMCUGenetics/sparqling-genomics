\documentclass[11pt,aspectratio=169]{beamer} % Add ,handout for printing
\usetheme{default}
\usecolortheme{UMCU}

\defbeamertemplate*{headline}{miniframes theme no subsection}
{%
  \begin{beamercolorbox}[colsep=1.5pt]{upper separation line head}
  \end{beamercolorbox}
  \begin{beamercolorbox}{section in head/foot}
    \vskip2pt\insertnavigation{\paperwidth}\vskip2pt
  \end{beamercolorbox}%
  \begin{beamercolorbox}[colsep=1.5pt]{lower separation line head}
  \end{beamercolorbox}
}
\setbeamertemplate{footline}[miniframes theme no subsection]

\usepackage[english]{babel}
\uselanguage{english}
\languagepath{english}

\usepackage{color}
\usepackage[utf8]{inputenc}
\usepackage{amsmath}
\usepackage{amsfonts}
\usepackage{amssymb}

\definecolor{CustomGray}{rgb}{0.3,0.3,0.3}
\definecolor{AlmostBlack}{rgb}{0.2,0.2,0.2}
\definecolor{DarkGreen}{rgb}{0.175,0.312,0.085}
\definecolor{DarkRed}{rgb}{0.312,0.086,0.086}

\author{}
\title{Using SPARQL and RDF to analyze structural variants}
\setbeamercovered{transparent}
\setbeamertemplate{navigation symbols}{}

\begin{document}
\beamertemplatenavigationsymbolsempty

\usebackgroundtemplate{%
  \includegraphics[width=\paperwidth]{images/presentation-background.pdf}
}
\begin{frame}

  { \color{white}
    \par
    \LARGE Using SPARQL and RDF to analyze structural variants
    \normalsize
    ~\\
    ~\\
    Roel Janssen
    ~\\
    ~\\
    \today}

%\titlepage
\end{frame}

\usebackgroundtemplate{%
  \includegraphics[width=\paperwidth]{images/slide-background.pdf}
}

%\begin{frame}
%\tableofcontents
%\end{frame}

\section{Introduction}

\subsection{Introduction}

\begin{frame}
  \frametitle{About structural variant calling}
  \begin{center}
    \includegraphics[width=0.60\textwidth]{images/variant-calling.pdf}
  \end{center}
\end{frame}

\begin{frame}
  \frametitle{Goals}

  \begin{itemize}
    \item Filter structural variant (SV) calls by position overlap*
    \item Filter or augment SV call information with regional information
  \end{itemize}

  \small{* Idea by Mark van Roosmalen and Robert Ernst}

\end{frame}

\begin{frame}
  \frametitle{RDF and SPARQL}
  \begin{itemize}
    \item Resource Description Framework (RDF)
      \begin{itemize}
        \item is an information modeling method;
        \item is a W3C recommendation since 1999;
        \item EMBL-EBI made data accessible in RDF format.
        \end{itemize}
    \item SPARQL Protocol and RDF Query Language (SPARQL)
      \begin{itemize}
        \item is a language to query data in RDF format;
        \item can be used in various programming languages (R, Python, Perl, ...).
      \end{itemize}
  \end{itemize}
\end{frame}

% \begin{frame}
%   \frametitle{Goals}
%   \begin{itemize}
%     \item Filter structural variant calls by intersecting datasets
%     \item Integrate with other data sources
%   \end{itemize}
% \end{frame}

%\subsection{Describing information with RDF}

\begin{frame}
  \frametitle{Describing information using RDF}
  \begin{center}
    \includegraphics[width=0.60\textwidth]{images/triple-pattern.pdf}
  \end{center}
\end{frame}

\begin{frame}
  \frametitle{Describing information using RDF}
  \begin{center}
    \includegraphics[width=0.60\textwidth]{images/triple-pattern-2.pdf}
  \end{center}
\end{frame}

% \begin{frame}
%   \frametitle{Describing information using RDF}
%   \begin{center}
%     \includegraphics[width=0.60\textwidth]{images/triple-pattern-3.pdf}
%   \end{center}
% \end{frame}

\section{Applying it to SVs}

\subsection{Applying it to SVs}

\begin{frame}
  \frametitle{Model: Extract triples from the Variant Call Format (VCF)}
  \begin{center}
    \includegraphics[height=0.80\textheight]{images/close-to-home.pdf}
  \end{center}
\end{frame}

\begin{frame}
  \frametitle{Tools: Extract triples from the Variant Call Format (VCF)}
  \begin{center}
    \includegraphics[height=0.70\textheight]{images/theplan.pdf}
  \end{center}
  Source code for the \texttt{vcf2turtle}: \url{https://github.com/UMCUgenetics/sparqling-svs}
\end{frame}

\begin{frame}
  \frametitle{Tools: Extract triples from the Variant Call Format (VCF)}
  \begin{center}
    \includegraphics[height=0.70\textheight]{images/theplan-2.pdf}
  \end{center}
  Source code for the \texttt{vcf2turtle}: \url{https://github.com/UMCUgenetics/sparqling-svs}
\end{frame}

\begin{frame}
  \frametitle{Tools: Query and ontology interface for quick exploration}
  \begin{columns}
    \begin{column}{0.48\textwidth}
      \includegraphics[width=1.00\textwidth]{images/query-editor1.png}
    \end{column}
    \begin{column}{0.48\textwidth}
      \begin{center}
        \includegraphics[width=1.00\textwidth]{images/ontology-viewer.png}~\\
        \includegraphics[width=1.00\textwidth]{images/ontology-viewer2.png}
        \end{center}
    \end{column}
\end{columns}
\end{frame}

\begin{frame}
  \frametitle{Filtering overlap}
  \begin{center}
    \includegraphics[width=0.85\textwidth]{images/filter-pattern.pdf}
  \end{center}
\end{frame}


\begin{frame}
  \frametitle{Ensembl gene regions}
  From \url{<http://rdf.ebi.ac.uk/resource/ensembl>}:
  \begin{center}
    \includegraphics[width=0.70\textwidth]{images/ensembl.pdf}
  \end{center}
\end{frame}

\begin{frame}
  \frametitle{Linking Ensembl gene regions with our SVs}
  \begin{center}
    \includegraphics[width=0.80\textwidth]{images/close-to-home-positions-1.pdf}
  \end{center}
\end{frame}

\begin{frame}
  \frametitle{Linking Ensembl gene regions with our SVs}
  \begin{center}
    \includegraphics[width=0.80\textwidth]{images/close-to-home-positions-2.pdf}
  \end{center}
\end{frame}

% \begin{frame}
%   \frametitle{Example 1: Gathering numbers on the database size}
%   Get the number of files, structural variants, and number of positions.

% \begin{example}
% PREFIX : \texttt{<http://localhost:5000/cth/>}\\
% ~\\
% \texttt{SELECT COUNT(DISTINCT ?origin) as ?numberOfSources}\\
% \texttt{       COUNT(DISTINCT ?variant) as ?numberOfSVs}\\
% \texttt{       COUNT(DISTINCT ?position) as ?numberOfPositions \{}\\
% %\texttt{WHERE \{}\\
% \texttt{  ?origin a :Origin .}\\
% \texttt{  ?variant a :StructuralVariant .}\\
% \texttt{  ?variant :genome\_position ?position .}\\
% \texttt{\}}
% \end{example}
% \end{frame}

\section{Wrapping up}

\begin{frame}
  \frametitle{Wrapping up}
  \begin{itemize}
    \item Triple stores scale by communicating with other triple stores;
    \pause
    \item By describing your data using RDF, you can tap into other databases;
    \pause
    \item ... and others could tap into yours (if you publish it);
    \pause
    \item Linking with other databases needs to be driven by research questions;
      \begin{itemize}
        \item Create a model to answer specific questions;
        \pause
        \item Don't over-engineer it.
      \end{itemize}
      \pause
    \item Slides will be available at: \url{https://github.com/UMCUGenetics/sparqling-svs}
  \end{itemize}
\end{frame}

\section{Acknowledgements}

\subsection{Acknowledgements}

\begin{frame}
  \frametitle{Acknowledgements}
  \begin{itemize}
    \item[] Arne Hoeck
    \item[] Arnold Kuzniar
    \item[] Joep de Ligt
    \item[] Mark van Roosmalen
    \item[] Robert Ernst
    \item[] Edwin Cuppen
  \end{itemize}
\end{frame}

%% \subsection{Functional package management}

%% \begin{frame}
%%   \frametitle{Functional package management}
%%   \begin{center}
%%   \includegraphics[height=0.75\textheight]{images/functional-package-management-1.pdf}
%%   \end{center}
%% \end{frame}

%% \begin{frame}
%%   \frametitle{Functional package management}
%%   \begin{center}
%%   \includegraphics[height=0.75\textheight]{images/functional-package-management-2.pdf}
%%   \end{center}
%% \end{frame}

%% \begin{frame}
%%   \frametitle{Functional package management}
%%   \begin{center}
%%   \includegraphics[height=0.75\textheight]{images/functional-package-management-3.pdf}
%%   \end{center}
%% \end{frame}

\end{document}
