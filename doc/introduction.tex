\chapter{Introduction}

\section{Prerequisites}
\label{sec:prerequisites}

  In addition to the tools provided by this project, a triple store is required.
  In the manual we use \href{https://virtuoso.openlinksw.com/}{Virtuoso}, but
  \href{https://github.com/4store/4store}{4store},
  \href{https://www.blazegraph.com/}{BlazeGraph}, or
  \href{https://allegrograph.com/}{AllegroGraph} may also be used.

  Before we can use the programs provided by this project, we need to build
  them first.

  The build system needs \href{https://www.gnu.org/software/autoconf}{GNU Autoconf},
  \href{https://www.gnu.org/software/automake}{GNU Automake},
  \href{https://www.gnu.org/software/make}{GNU Make} and
  \href{https://www.freedesktop.org/wiki/Software/pkg-config/}{pkg-config}.
  Additionally, for building the documentation, a working \LaTeX{} distribution is
  required including the \texttt{pdflatex} program.  Because \LaTeX{} distributions
  are rather large, this is optional.

  Each component in the repository has its own dependencies.  Table
  \ref{table:dependencies} provides an overview for each tool.

  \hypersetup{urlcolor=black}
  \begin{table}[H]
    \begin{tabularx}{\textwidth}{ X X X }
      \headrow
      \textbf{vcf2turtle} & \textbf{Web interface} & \textbf{Documentation}\\
      \evenrow
      \href{https://gcc.gnu.org/}{GNU C compiler}
      & \href{https://www.gnu.org/software/guile}{GNU Guile}
      & \href{https://tug.org/texlive/}{\LaTeX{} distribtion}\\
      \oddrow
      \href{https://www.gnupg.org/related_software/libgcrypt/}{libgcrypt}
      & 
      & \\
      \evenrow
      \href{http://www.htslib.org/}{HTSLib}
      &
      & \\
    \end{tabularx}
    \caption{\small External tools required to build and run the programs this
      project provides.}
    \label{table:dependencies}
  \end{table}
  \hypersetup{urlcolor=LinkGray}

  We suggest \href{https://curl.haxx.se/}{cURL} to import RDF to a triple
  store.  The manual provides example commands to import RDF using cURL.

\section{Setting up a build environment}

\subsection{Debian}

  Debian includes all tools, so use this command to install the
  build dependencies:

\begin{siderules}
\begin{verbatim}
apt-get install autoconf automake gcc make pkg-config libgcrypt-dev \
                zlib-dev guile-2.0 guile-2.0-dev texlive curl
\end{verbatim}
\end{siderules}

  The command has been tested on Debian 9.

\subsection{CentOS}

  CentOS 7 does not include \texttt{htslib}.  All other dependencies can
  be installed using the following command:

\begin{siderules}
\begin{verbatim}
yum install autoconf automake gcc make pkgconfig libgcrypt-devel \
            guile guile-devel texlive curl
\end{verbatim}
\end{siderules}

\subsection{GNU Guix}

  If \href{https://www.gnu.org/software/guix}{GNU Guix} is available on your
  system, an environment that contains all external tools required to build
  the programs in this project can be obtained running the following command
  from the project's repository root:

\begin{siderules}
\begin{verbatim}
guix environment -l guix.scm
\end{verbatim}
\end{siderules}

\section{Installation instructions}

After installing the required tools (see section \ref{sec:prerequisites}),
building involves running the following commands:
\begin{siderules}
\begin{verbatim}
autoreconf -vfi && ./configure
make && make install
\end{verbatim}
\end{siderules}

Alternatively, the individual components can be built by replacing
\texttt{make \&\& make install} with \texttt{make -C <component-directory>}.
So, to only build \texttt{vcf2turtle}, the following command could be used:
\begin{siderules}
\begin{verbatim}
make -C tools/vcf2turtle
\end{verbatim}
\end{siderules}
