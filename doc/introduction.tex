\chapter{Getting started}

  SPARQLing genomics is a combination of tools and practices to create a
  knowledge graph to make \i{discovering}, \i{connecting}, and
  \i{collaborating} easy.

\section{Prerequisites}
\label{sec:prerequisites}

  The programs provided by this project are designed to build a knowledge graph.
  However, a knowledge graph store (better known as an RDF store) is not included
  because various great RDF stores already exist, including
  \href{https://virtuoso.openlinksw.com/}{Virtuoso},
  \href{https://github.com/4store/4store}{4store} and
  \href{https://www.blazegraph.com/}{BlazeGraph}.  We recommend using one of
  the mentioned RDF stores with the programs from this project.

  Before we can use the programs provided by this project, we need to build
  them.  The build system needs
  \href{https://www.gnu.org/software/autoconf}{GNU Autoconf},
  \href{https://www.gnu.org/software/automake}{GNU Automake},
  \href{https://www.gnu.org/software/make}{GNU Make} and
  \href{https://www.freedesktop.org/wiki/Software/pkg-config/}{pkg-config}.
  Additionally, for building the documentation, a working \LaTeX{} distribution is
  required including the \t{pdflatex} program.  Because \LaTeX{} distributions
  are rather large, this is dependency is optional, at the cost of not being able
  to (re)generate the documentation.

  Each component in the repository has its own dependencies.  Table
  \ref{table:dependencies} provides an overview for each tool.  A \B{}
  indicates that the program (row) depends on the program or library (column).
  Care was taken to pick dependencies that are widely available on GNU/Linux
  systems.

  \hypersetup{urlcolor=black}
  \begin{table}[H]
    \begin{tabularx}{\textwidth}{X *{9}{!{\color{white}\VRule[1pt]}l}}
      \headrow \cellcolor{White}
      & \rotatebox[origin=l]{90}{\href{https://gcc.gnu.org/}{C compiler}\space\space\space}
      & \rotatebox[origin=l]{90}{\href{http://www.librdf.org/}{raptor2}}
      & \rotatebox[origin=l]{90}{\href{http://www.xmlsoft.org/}{libxml2}}
      & \rotatebox[origin=l]{90}{\href{http://www.htslib.org/}{HTSLib}}
      & \rotatebox[origin=l]{90}{\href{https://zlib.net/}{zlib}}
      & \rotatebox[origin=l]{90}{\href{https://www.gnu.org/software/guile}{GNU Guile}}
      & \rotatebox[origin=l]{90}{\href{https://www.gnutls.org/}{GnuTLS}}
      & \rotatebox[origin=l]{90}{\href{https://tug.org/texlive/}{\LaTeX{}}}\\
      \evenrow
      \t{vcf2rdf}         & \B & \B &    & \B &    &    & \B &\\
      \oddrow
      \t{bam2rdf}         & \B & \B &    & \B &    &    & \B &\\
      \evenrow
      \t{table2rdf}       & \B & \B &    &    & \B &    & \B &\\
      \oddrow
      \t{json2rdf}        & \B & \B &    &    & \B &    & \B &\\
      \evenrow
      \t{xml2rdf}         & \B & \B & \B &    & \B &    & \B &\\
      \oddrow
      \t{folder2rdf}      &    &    &    &    &    & \B &    &\\
      \evenrow
      \t{sg-web}          & \B &    &    &    &    & \B & \B &\\
      \oddrow
      \t{sg-web-test}     &    &    &    &    &    & \B & \B &\\
      \evenrow
      \t{sg-auth-manager} &    &    &    &    &    & \B & \B &\\
      \oddrow
      Documentation       &    &    &    &    &    &    &    & \B \\
    \end{tabularx}
    \caption{\small External tools required to build and run the programs of
      this project.}
    \label{table:dependencies}
  \end{table}
  \hypersetup{urlcolor=LinkGray}

  The manual provides example commands to import RDF using
  \href{https://curl.haxx.se/}{cURL}.

\section{Installing the prerequisites}

\subsection{Debian}

  Debian includes all tools, so use this command to install the
  build dependencies:

\begin{siderules}
\begin{verbatim}
apt-get install autoconf automake gcc make pkg-config zlib1g-dev  \
                guile-2.2 guile-2.2-dev libraptor2-dev libhts-dev \
                texlive curl libxml2-dev gnutls-dev
\end{verbatim}
\end{siderules}

  This command has been tested on Debian 10.  If you're using a different
  version of Debian, some package names may differ.

\subsection{CentOS}

  CentOS 7 and 8 do not include \t{htslib}.  Follow the instructions on
  the \href{https://www.htslib.org/}{\t{htslib} website}%
  \footnote{https://www.htslib.org/} to build \t{htslib} from source.

  All other dependencies can be installed using the following command:

\begin{siderules}
\begin{verbatim}
yum install autoconf automake gcc make pkgconfig guile guile-devel \
            raptor2-devel texlive curl libxml2-devel gnutls-devel
\end{verbatim}
\end{siderules}

\subsection{GNU Guix}

  For GNU Guix, use the \t{environment.scm} file to set up the development
  environment:

\begin{siderules}
\begin{verbatim}
guix environment -l environment.scm
\end{verbatim}
\end{siderules}

\subsection{MacOS}

  The necessary dependencies to build \t{sparqling-genomics} can be
  installed using \href{https://brew.sh/}{homebrew}:

\begin{siderules}
\begin{verbatim}
brew install autoconf automake gcc make pkg-config guile \
             htslib curl raptor libxml2 zlib gnutls
\end{verbatim}
\end{siderules}

  Due to a missing \LaTeX{} distribution on MacOS, the documentation
  cannot be build.

\subsection{Microsoft Windows}

  For those who embrace Microsoft Windows, one can extend it with the
  dependencies for SPARQLing-genomics in two ways: Either use the Windows
  Subsystem for Linux and use the instructions for GNU/Linux, or use
  \href{https://www.msys2.org/}{MSYS2}\footnote{\url{https://www.msys2.org/}}.
  After installing MSYS2, issue the following command from its console:

\begin{siderules}
\begin{verbatim}
pacman -S autoconf automake gcc make pkg-config libguile-devel guile curl \
          libxml2 zlib gnutls
\end{verbatim}
\end{siderules}

  There are two missing dependencies: \t{raptor2} -- for which a package is
  \href{https://github.com/msys2/MINGW-packages/tree/master/mingw-w64-raptor2}{upcoming}%
  \footnote{\url{https://github.com/msys2/MINGW-packages/tree/master/mingw-w64-raptor2}},
  and \t{htslib} -- for which installation instructions can be found in a
  \href{https://github.com/samtools/htslib/issues/907}{Github issue of
    \t{htslib}}\footnote{\url{https://github.com/samtools/htslib/issues/907}}.

\section{Obtaining the source code}
\label{sec:obtaining-tarball}

  \begin{sloppypar}
  The source code can be downloaded at the
  \href{https://github.com/UMCUGenetics/sparqling-genomics/releases}%
  {Releases}%
  \footnote{\url{https://github.com/UMCUGenetics/sparqling-genomics/releases}}
  page.  Make sure to download the {\fontfamily{\ttdefault}\selectfont
    sparqling-genomics-\sgversion{}.tar.gz} file.
  \end{sloppypar}

  Or, directly download the tarball using the command-line:
\begin{siderules}
\begin{lstlisting}[language=bash]
curl -LO https://github.com/UMCUGenetics/sparqling-genomics/releases/\
download/(@*\sgversion{}*@)/sparqling-genomics-(@*\sgversion{}*@).tar.gz
\end{lstlisting}
\end{siderules}

  After obtaining the tarball, it can be unpacked using the \t{tar}
  command:

\begin{siderules}
\begin{lstlisting}
tar zxvf sparqling-genomics-(@*\sgversion{}*@).tar.gz
\end{lstlisting}
\end{siderules}

\section{Installation instructions}

  After installing the required tools (see section \refer{sec:prerequisites}),
  and obtaining the source code (see section \refer{sec:obtaining-tarball}),
  building involves running the following commands:

\begin{siderules}
\begin{lstlisting}
cd sparqling-genomics-(@*\sgversion{}*@)
autoreconf -vif # Only needed if the "./configure" step does not work.
./configure
make
make install
\end{lstlisting}
\end{siderules}

  To run the \t{make install} command, super user privileges may be
  required.  This step can be ignored, but will keep the tools in the project's
  directory.  So in that case, invoking \t{vcf2rdf} must be done using
  \t{tools/vcf2rdf/vcf2rdf} when inside the project's root directory,
  instead of ``just'' \t{vcf2rdf}.

  Alternatively, specify a \t{-{}-prefix} to the \t{configure}
  script to install the tools to a user-writeable location.

  Individual components can be built by replacing \t{make} with the
  more specific \t{make -C <component-directory>}.  So, to \i{only}
  build \t{vcf2rdf}, the following command could be used:

\begin{siderules}
\begin{verbatim}
make -C tools/vcf2rdf
\end{verbatim}
\end{siderules}

\section{Using a pre-built Docker image}

  A pre-built Docker container can be obtained from the release page.  It
  can be imported into docker using the following commands:

\begin{siderules}
\begin{lstlisting}
curl -LO https://github.com/UMCUGenetics/sparqling-genomics/releases/\
download/(@*\sgversion{}*@)/sparqling-genomics-(@*\sgversion{}*@)-docker.tar.gz
docker load < sparqling-genomics-(@*\sgversion{}*@)-docker.tar.gz
\end{lstlisting}
\end{siderules}

  The container includes both SPARQLing genomics and Virtuoso (open source
  edition).
