\chapter{Getting started}

  SPARQLing genomics is a combination of tools and practices to create a
  knowledge graph to make \i{discovering}, \i{connecting}, and
  \i{collaborating} easy.

  The programs provided by this project are designed to build a knowledge graph.
  However, a knowledge graph store (better known as an RDF store) is not included
  because various great RDF stores already exist, including
  \href{https://virtuoso.openlinksw.com/}{Virtuoso},
  \href{https://github.com/4store/4store}{4store} and
  \href{https://www.blazegraph.com/}{BlazeGraph}.  We recommend using one of
  the mentioned RDF stores with the programs from this project.

  There are two ways to install SPARQLing genomics; using pre-built binaries
  on supported GNU/Linux distributions, or by building it from its from source
  code.

\section{Binary installation}
\label{sec:binary-installation}

\subsection{Debian}

  Write the following line to \file{/etc/apt/sources.list.d/sparqling-genomics.list}:

%% TODO: Set up package signing, and remove the [trusted=yes] bits.

\begin{lstlisting}
deb [trusted=yes] https://debian.sparqling-genomics.org buster main
\end{lstlisting}

  This will add a repository that contains the \code{sparqling-genomics}
  package.  To install it, use the following commands:

\begin{lstlisting}[language=bash]
apt-get update
apt-get install sparqling-genomics
\end{lstlisting}

\subsection{CentOS}

  CentOS 7 and 8 do not include \t{htslib} by default.  Enable the
  \href{https://fedoraproject.org/wiki/EPEL}{EPEL repository}%
  \footnote{https://fedoraproject.org/wiki/EPEL} to make sure the
  \t{htslib} package is found while resolving dependencies for
  \t{sparqling-genomics}.

  With the EPEL repository set up, write the following to
  \file{/etc/yum.repos.d/sparqling-genomics.repo}:

\begin{lstlisting}
[sparqling-genomics]
name=SPARQLing-genomics
baseurl=https://rpm.sparqling-genomics.org/centos
enabled=1
gpgcheck=1
gpgkey=https://rpm.sparqling-genomics.org/sparqling-genomics.key
\end{lstlisting}

This will add a repository that contains the \code{sparqling-genomics} package.
To install it, use the following commands:

\begin{lstlisting}[language=bash]
yum update
yum install sparqling-genomics
\end{lstlisting}

\subsection{Fedora}

Write the following to \file{/etc/yum.repos.d/sparqling-genomics.repo}:

\begin{lstlisting}
[sparqling-genomics]
name=SPARQLing-genomics
baseurl=https://rpm.sparqling-genomics.org/fedora
enabled=1
gpgcheck=1
gpgkey=https://rpm.sparqling-genomics.org/sparqling-genomics.key
\end{lstlisting}

This will add a repository that contains the \code{sparqling-genomics} package.
To install it, use the following commands:

\begin{lstlisting}
dnf update
dnf install sparqling-genomics
\end{lstlisting}

\subsection{Generic tarball for GNU/Linux}

  A self-contained distribution of SPARQLing-genomics and its dependencies
  can be obtained from the \href{https://github.com/UMCUGenetics/sparqling-genomics/releases}%
  {release page}.  All programs are ready-to-run after unpacking the tarball.

\begin{lstlisting}
curl -LO https://github.com/UMCUGenetics/sparqling-genomics/releases/\
download/(@*\sgversion{}*@)/sparqling-genomics-(@*\sgversion{}*@)-pack.tar.gz
tar axvf sparqling-genomics-(@*\sgversion{}*@)-pack.tar.gz
bin/table2rdf --help
\end{lstlisting}

  The tarball includes both SPARQLing genomics and Virtuoso (open source
  edition).

  The downsides of using the generic tarball in contrast to the repositories is
  that the software will not receive updates, and the dependencies are not
  shared between your distribution and SPARQLing genomics.

\subsection{Docker image}

  A pre-built Docker container can be obtained from the release page.  It
  can be imported into docker using the following commands:

\begin{lstlisting}
curl -LO https://github.com/UMCUGenetics/sparqling-genomics/releases/\
download/(@*\sgversion{}*@)/sparqling-genomics-(@*\sgversion{}*@)-docker.tar.gz
docker load < sparqling-genomics-(@*\sgversion{}*@)-docker.tar.gz
\end{lstlisting}

  The container includes both SPARQLing genomics and Virtuoso (open source
  edition).

\pagebreak{}
\section{Install from source}
\label{sec:source-installation}

  Each component in the repository has its own dependencies.  Table
  \ref{table:dependencies} provides an overview for each tool.  A \B{}
  indicates that the program (row) depends on the program or library (column).
  Care was taken to pick dependencies that are widely available on GNU/Linux
  systems.

  \hypersetup{urlcolor=black}
  \begin{table}[H]
    \begin{tabularx}{\textwidth}{X *{9}{!{\color{white}\VRule[1pt]}l}}
      \headrow \cellcolor{White}
      & \rotatebox[origin=l]{90}{\href{https://gcc.gnu.org/}{C compiler}\space\space\space}
      & \rotatebox[origin=l]{90}{\href{http://www.librdf.org/}{raptor2}}
      & \rotatebox[origin=l]{90}{\href{http://www.xmlsoft.org/}{libxml2}}
      & \rotatebox[origin=l]{90}{\href{http://www.htslib.org/}{HTSLib}}
      & \rotatebox[origin=l]{90}{\href{https://zlib.net/}{zlib}}
      & \rotatebox[origin=l]{90}{\href{https://www.gnu.org/software/guile}{GNU Guile}}
      & \rotatebox[origin=l]{90}{\href{https://www.gnutls.org/}{GnuTLS}}
      & \rotatebox[origin=l]{90}{\href{http://www.libpng.org/pub/png/libpng.html}{libpng}}
      & \rotatebox[origin=l]{90}{\href{https://tug.org/texlive/}{\LaTeX{}}}\\
      \evenrow
      \t{vcf2rdf}         & \B & \B &    & \B &    &    & \B &    &\\
      \oddrow
      \t{bam2rdf}         & \B & \B &    & \B &    &    & \B &    &\\
      \evenrow
      \t{table2rdf}       & \B & \B &    &    & \B &    & \B &    &\\
      \oddrow
      \t{json2rdf}        & \B & \B &    &    & \B &    & \B &    &\\
      \evenrow
      \t{xml2rdf}         & \B & \B & \B &    & \B &    & \B &    &\\
      \oddrow
      \t{folder2rdf}      &    &    &    &    &    & \B &    &    &\\
      \evenrow
      \t{sg-web}          & \B &    &    &    &    & \B & \B & \B &\\
      \oddrow
      \t{sg-web-test}     & \B &    &    &    &    & \B & \B &    &\\
      \evenrow
      \t{sg-auth-manager} & \B &    &    &    &    & \B & \B &    &\\
      \oddrow
      Documentation       &    &    &    &    &    &    &    &    & \B \\
    \end{tabularx}
    \caption{\small External tools required to build and run the programs of
      this project.}
    \label{table:dependencies}
  \end{table}
  \hypersetup{urlcolor=LinkGray}

  The manual provides example commands to import RDF using
  \href{https://curl.haxx.se/}{cURL}.

  The \program{sg-web} program contains an embedded version of
  \href{https://github.com/libharu/libharu}{libharu} to generate PDFs.

\subsection{Optional dependencies and features}

  A fully operational version of SPARQLing-genomics can be built with the
  dependencies from table \ref{table:dependencies}.  To integrate into specific
  set-ups, and enable extending SPARQLing-genomics with other programming
  languages, some additional dependencies may be required.  This section
  describes the optional dependencies required for specific set-ups.

\subsubsection{Integrating \program{sg-web} in an LDAP environment}

  User management in \program{sg-web} can be offloaded to LDAP (version 3).
  For this feature, the \program{openldap} package and C libraries needs to
  specified using the \t{-{}-with-libldap-prefix} switch passed to the
  \t{configure} script.

\subsubsection{Enabling the virtual filesystem}

  To build \program{sgfs}, \t{fuse} must be available when the \t{configure}
  script is run.

\subsubsection{Extending \program{sg-web} with R}

  In addition to C and Scheme, reports can be written in R.  This requires
  \t{libR} (shipped with the default R installation) to be available at
  compile-time.

\section{Installing the prerequisites}

\subsection{Debian}

  Use the following command to install the build dependencies:

\begin{lstlisting}
apt-get install autoconf automake gcc make pkg-config zlib1g-dev  \
                guile-2.2 guile-2.2-dev libraptor2-dev libhts-dev \
                texlive curl libxml2-dev libgnutls28-dev libpng-dev
\end{lstlisting}

  This command has been tested on Debian 10.  If you're using a different
  version of Debian, some package names may differ.

\subsection{CentOS}

  CentOS 7 and 8 do not include \t{htslib}.  Either enable the
  \href{https://fedoraproject.org/wiki/EPEL}{EPEL repository}%
  \footnote{https://fedoraproject.org/wiki/EPEL}
  to install \t{htslib} or follow the instructions on the
  \href{https://www.htslib.org/}{\t{htslib} website}%
  \footnote{https://www.htslib.org/} to build \t{htslib} from source.

  All other dependencies can be installed using the following command:

\begin{lstlisting}
yum install autoconf automake gcc make pkgconfig guile guile-devel \
            raptor2-devel texlive curl libxml2-devel gnutls-devel  \
            libpng-devel
\end{lstlisting}

\subsection{GNU Guix}

  For GNU Guix, use the \t{guix.scm} file to set up the development
  environment:

\begin{lstlisting}
guix environment -l guix.scm
\end{lstlisting}

\subsection{macOS}

  For macOS, the necessary dependencies to build SPARQLing-genomics
  can be installed using
  \href{https://brew.sh/}{homebrew}\footnote{\url{https://brew.sh/}}:

\begin{lstlisting}
brew install autoconf automake gcc make pkg-config guile \
             htslib curl raptor libxml2 zlib gnutls libpng
\end{lstlisting}

  Due to a missing \LaTeX{} distribution on MacOS, the documentation
  cannot be build.

\subsection{Microsoft Windows}

  For those who embrace Microsoft Windows, one can extend it with the
  dependencies for SPARQLing-genomics in two ways: Either use the Windows
  Subsystem for Linux and use the instructions for GNU/Linux, or use
  \href{https://www.msys2.org/}{MSYS2}\footnote{\url{https://www.msys2.org/}}.
  After installing MSYS2, issue the following command from its console:

\begin{lstlisting}
pacman -S autoconf automake gcc make pkg-config libguile-devel guile curl \
          libxml2 zlib gnutls
\end{lstlisting}

  There are two missing dependencies: \t{raptor2} -- for which a package is
  \href{https://github.com/msys2/MINGW-packages/tree/master/mingw-w64-raptor2}{upcoming}%
  \footnote{\url{https://github.com/msys2/MINGW-packages/tree/master/mingw-w64-raptor2}},
  and \t{htslib} -- for which installation instructions can be found in a
  \href{https://github.com/samtools/htslib/issues/907}{Github issue of
    \t{htslib}}\footnote{\url{https://github.com/samtools/htslib/issues/907}}.

\section{Obtaining the source code}
\label{sec:obtaining-tarball}

  \begin{sloppypar}
  The source code can be downloaded at the
  \href{https://github.com/UMCUGenetics/sparqling-genomics/releases}%
  {Releases}%
  \footnote{\url{https://github.com/UMCUGenetics/sparqling-genomics/releases}}
  page.  Make sure to download the {\fontfamily{\ttdefault}\selectfont
    sparqling-genomics-\sgversion{}.tar.gz} file.
  \end{sloppypar}

  Or, directly download the tarball using the command-line:

\begin{lstlisting}[language=bash]
curl -LO https://github.com/UMCUGenetics/sparqling-genomics/releases/\
download/(@*\sgversion{}*@)/sparqling-genomics-(@*\sgversion{}*@).tar.gz
\end{lstlisting}

  After obtaining the tarball, it can be unpacked using the \t{tar}
  command:

\begin{lstlisting}
tar zxvf sparqling-genomics-(@*\sgversion{}*@).tar.gz
\end{lstlisting}

\section{Installation instructions}

  After installing the required tools (see section
  \refer{sec:source-installation}), and obtaining the source code (see
  section \refer{sec:obtaining-tarball}), building involves running
  the following commands:

\begin{lstlisting}
cd sparqling-genomics-(@*\sgversion{}*@)
autoreconf -vif # Only needed if the "./configure" step does not work.
./configure
make
make install
\end{lstlisting}

  To run the \t{make install} command, super user privileges may be
  required.  This step can be ignored, but will keep the tools in the project's
  directory.  So in that case, invoking \t{vcf2rdf} must be done using
  \t{tools/vcf2rdf/vcf2rdf} when inside the project's root directory,
  instead of ``just'' \t{vcf2rdf}.

  Alternatively, specify a \t{-{}-prefix} to the \t{configure}
  script to install the tools to a user-writeable location.

  Individual components can be built by replacing \t{make} with the
  more specific \t{make -C <component-directory>}.  So, to \i{only}
  build \t{vcf2rdf}, the following command could be used:

\begin{lstlisting}
make -C tools/vcf2rdf
\end{lstlisting}
