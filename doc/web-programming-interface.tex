\section{Programming interface}
\label{sec:web-api}

  Other than a user interface, the web interface provides a programming interface
  using the HyperText Transport Protocol (HTTP).  The interface supports XML,
  JSON, and S-expressions.  Table \ref{table:api-return-formats} summarizes the
  supported formats.

  \hypersetup{urlcolor=black}
  \begin{table}[H]
    \begin{tabularx}{\textwidth}{l!{\VRule[-1pt]}X}
      \headrow
      \b{Content-Type} & \b{Example response}\\
      \evenrow
      \t{application/json}
      & \t{[\{ "message": "This is a JSON response." \}]}\\
      \oddrow
      \t{application/xml}
      & \t{<message>This is a XML response.</message>}\\
      \evenrow
      \t{application/s-expression}
      & \t{(message . "This is a S-expression response.")}\\
    \end{tabularx}
    \caption{\small Implemented content types for the API.  The
      \t{Content-Type} can be used in the \t{Accept} HTTP header.}
    \label{table:api-return-formats}
  \end{table}
  \hypersetup{urlcolor=LinkGray}

\subsection{Formatting \t{POST} requests}

  The \t{Accept} parameter influences the response format of the API,
  and the \t{Content-Type} parameter can be used to indicate the format
  of the request.

  In addition to the documented \t{Content-Type} values, the type
  \t{application/x-www-form-urlencoded} is also allowed, which expects
  the following format:
\begin{siderules}
\begin{verbatim}
parameter1=value1&parameter2=value2&...
\end{verbatim}
\end{siderules}

  The \t{Content-Type} header line does not have to be equal to the
  \t{Accept} header line, so for example, parameters can be sent in
  XML and the response can be formatted in JSON or the other way around.

\subsection{Conventions when using XML}

\begin{sloppypar}
  When using XML, there are a few conventions to follow.  To communicate a list
  or array of records, the API relies on using pre-defined elements. The API
  expects parameters to be wrapped in \t{<parameters>...</parameters>}
  elements.  So, to log in using an XML request, the following message will be
  accepted:
\end{sloppypar}

\begin{siderules}
\begin{verbatim}
POST /api/login HTTP/1.1
Host: ...
Content-Type: application/xml
Content-Length: 80
Connection: close

<parameters>
  <username>...</username>
  <password>...</password>
</parameters>
\end{verbatim}
\end{siderules}

  Subsequently, when \t{Accept}ing XML the results are wrapped in
  \t{<results>...</results>}, and data structures built from multiple
  key-value pairs are wrapped in \t{<result>...<result>}.  The
  following example illustrates receiving a project record:

\begin{siderules}
\begin{verbatim}
GET /api/projects HTTP/1.1
Host: ...
Accept: application/xml
Cookie: ...
Connection: close

<results>
  <result>
    <projectId>http://sparqling-genomics.org/0.99.10/Project/...</projectId>
    <creator>http://sparqling-genomics.org/0.99.10/Agent/...</creator>
    ...
  </result>
</results>
\end{verbatim}
\end{siderules}

\b{Note}: The actual response strips the whitespace.  It was added to
this example for improved readability.

\subsection{Authenticating API requests with \t{/api/login}}
\label{sec:api-login}

  Before being able to interact with the API, a session token must be obtained.
  This can be done by sending a \t{POST}-request to \t{/api/login},
  with the following parameters:

  \hypersetup{urlcolor=black}
  \begin{table}[H]
    \begin{tabularx}{\textwidth}{l!{\VRule[-1pt]}X!{\VRule[-1pt]}X}
      \headrow
      \b{Parameter} & \b{Example} & \b{Required?}\\
      \evenrow
      \t{username}  & \t{jdoe}    & Yes\\
      \oddrow
      \t{password}  & \t{secret}  & Yes\\
    \end{tabularx}
  \end{table}
  \hypersetup{urlcolor=LinkGray}

  The following cURL command would log the user \t{jdoe} in:

\begin{siderules}
\begin{verbatim}
curl --cookie-jar - http://localhost/api/login \
     --data "username=jdoe&password=secret"
\end{verbatim}
\end{siderules}

  The response contains a cookie with the session token.  Use this cookie in
  subsequent requests.  When authentication fails, the service will respond
  with HTTP status-code \t{401}.

\subsection{Token-based authentication}
\label{sec:tokens}

  When automating data importing and running queries there comes the point
  at which you have a choice: Do I put in my login credentials in a script,
  a separate file, or anywhere else on the filesystem?  This bad security
  practice can be overcome by using \i{tokens}.

  A token is a randomly generated string that can be used to authenticate
  with, instead of your username and password.  Tokens are easy to create,
  and more importantly, easy to revoke.

  Generating a token can be done on the \i{Dashboard}, or using the API call
  \t{/api/new-session-token}.  Removing a token can be done on the
  \i{Dashboard}, or using the API call \t{/api/delete-token}.

  After generating a token, it can be used as a cookie in API calls.  The name
  of the cookie is always \t{SGSession}, regardless of the name of the
  token.

\subsection{Managing connections}

  The API can be used to store connections and refer to them by name.  The
  remainder of this section describes the API calls related to connections.

\subsubsection{Retrieve connections with \t{/api/connections}}

  With a call to \t{/api/connections}, the pre-configured connections
  can be viewed.  This resource expects a \t{GET} request and needs no
  parameters.  It returns all connection records associated with the currently
  logged-in user.

  The following cURL command would retrieve all connection records in JSON
  format:

\begin{siderules}
\begin{verbatim}
curl http://localhost/api/connections \
     --cookie "$COOKIE"               \
     -H "Accept: application/json"
\end{verbatim}
\end{siderules}

\subsubsection{Create a new connection with \t{/api/add-connection}}
\label{sec:api-create-connection}

  A call to \t{/api/add-connection} will create a new connection.
  The table below summarizes the parameters that can be used in this call.

  \hypersetup{urlcolor=black}
  \begin{table}[H]
    \begin{tabularx}{\textwidth}{l!{\VRule[-1pt]}X!{\VRule[-1pt]}X}
      \headrow
      \b{Parameter} & \b{Example}            & \b{Required?}\\
      \evenrow
      \t{name}      & \t{Example}            & Yes\\
      \oddrow
      \t{uri}       & \t{http://my/endpoint} & Yes\\
      \evenrow
      \t{username}  & \t{jdoe}               & No\\
      \oddrow
      \t{password}  & \t{secret}             & No\\
      \evenrow
      \t{backend}   & \t{4store}             & No\\
    \end{tabularx}
  \end{table}
  \hypersetup{urlcolor=LinkGray}

  The following cURL command would create a connection, using JSON as
  the format to express the parameters:

\begin{siderules}
\begin{verbatim}
curl http://localhost/api/add-connection          \
     --cookie "$COOKIE"                           \
     -H "Accept: application/xml"                 \
     -H "Content-Type: application/json"          \
     --data '{ "name":     "Example",             \
                "uri":      "http://my/endpoint", \
                "username": "jdoe",               \
                "password": "secret",             \
                "backend":  "4store" }'
\end{verbatim}
\end{siderules}

\subsubsection{Remove a connection with \t{/api/remove-connection}}

  A call to \t{/api/remove-connection} will remove an existing
  connection.  The table below summarizes the parameters that can be
  used in this call.

  \hypersetup{urlcolor=black}
  \begin{table}[H]
    \begin{tabularx}{\textwidth}{l!{\VRule[-1pt]}X!{\VRule[-1pt]}X}
      \headrow
      \b{Parameter} & \b{Example} & \b{Required?}\\
      \evenrow
      \t{name}      & \t{Example} & Yes\\
    \end{tabularx}
  \end{table}
  \hypersetup{urlcolor=LinkGray}

  The following cURL command would remove the connection that was created in
  section \refer{sec:api-create-connection}:

\begin{siderules}
\begin{verbatim}
curl http://localhost/api/remove-connection       \
     --cookie "$COOKIE"                           \
     -H "Accept: application/xml"                 \
     --data "name=Example"
\end{verbatim}
\end{siderules}

\subsection{Managing projects}

  The API has various resources to manage projects, as described in
  section \refer{sec:web-projects}.

\subsubsection{Retrieve a list of projects with \t{/api/projects}}

  Retrieving a list of projects can be done by sending a \t{GET} request
  to \t{/api/projects}.

  The following cURL command would retrieve a list of projects:

\begin{siderules}
\begin{verbatim}
curl http://localhost/api/projects      \
     --cookie "$COOKIE"                 \
     -H "Accept: application/xml"
\end{verbatim}
\end{siderules}

\subsubsection{Create a new project with \t{/api/add-project}}
\label{sec:api-add-project}

  To create a new project, send a \t{POST} request to
  \t{/api/add-project}.  The table below summarizes the parameters
  that can be used with this call.

  \hypersetup{urlcolor=black}
  \begin{table}[H]
    \begin{tabularx}{\textwidth}{l!{\VRule[-1pt]}X!{\VRule[-1pt]}X}
      \headrow
      \b{Parameter} & \b{Example}         & \b{Required?}\\
      \evenrow
      \t{name}      & \t{Example project} & Yes\\
    \end{tabularx}
  \end{table}
  \hypersetup{urlcolor=LinkGray}

  The following cURL command would add a project:

\begin{siderules}
\begin{verbatim}
curl http://localhost/api/add-project      \
     --cookie "$COOKIE"                    \
     -H "Accept: application/xml"          \
     --data "name=Example project"
\end{verbatim}
\end{siderules}

\subsubsection{Remove a project with \t{/api/remove-project}}

  To remove a project, send a \t{POST} request to
  \t{/api/remove-project}.  The table below summarizes the parameters
  that can be used with this call.

  \hypersetup{urlcolor=black}
  \begin{table}[H]
    \begin{tabularx}{\textwidth}{l!{\VRule[-1pt]}l!{\VRule[-1pt]}X}
      \headrow
      \b{Parameter}   & \b{Example} & \b{Required?}\\
      \evenrow
      \t{project-uri}
      & \t{http://sparqling-genomics.org/Project/640c0...5a6d2} & Yes\\
    \end{tabularx}
  \end{table}
  \hypersetup{urlcolor=LinkGray}

  Alternatively, instead of the full URI, the project's hash can be used.
  In that case, the parameters are:

  \hypersetup{urlcolor=black}
  \begin{table}[H]
    \begin{tabularx}{\textwidth}{l!{\VRule[-1pt]}l!{\VRule[-1pt]}X}
      \headrow
      \b{Parameter}    & \b{Example}       & \b{Required?}\\
      \evenrow
      \t{project-hash} & \t{640c0...5a6d2} & Yes\\
    \end{tabularx}
  \end{table}
  \hypersetup{urlcolor=LinkGray}

  The following cURL command would remove the project created in
  \refer{sec:api-add-project}:

\begin{siderules}
\begin{verbatim}
curl http://localhost/api/remove-project      \
     --cookie "$COOKIE"                       \
     -H "Accept: application/xml"             \
     --data "project-uri=http://sparqling-genomics.org/Project/640c0...5a6d2"
\end{verbatim}
\end{siderules}

\subsubsection{Assign a graph to a project \t{/api/assign-graph}}

  Assigning a graph to a project can be done by sending a \t{POST} request
  to \t{/api/assign-graph}.  The required parameters are:

  \begin{itemize}
    \item{\t{project-uri}: This can be obtained from a call to
      \t{/api/projects}.}
    \item{\t{connection}: The name of a connection obtained from
      a call to \t{/api/connections}.}
    \item{\t{graph-uri}: The graph URI to assign.}
  \end{itemize}

  The following example based on cURL shows how to perform the request:
\begin{siderules}
\begin{verbatim}
curl -X POST                                               \
     -H "Accept: application/json"                         \
     -H "Content-Type: application/json"                   \
     --cookie "SGSession=..."                              \
     --data '{ "project-uri": "http://the-project-uri",    \
               "connection": "wikidata",                   \
               "graph-uri": "http://the-new-graph-name" }' \
     http://localhost/api/assign-graph
\end{verbatim}
\end{siderules}

\subsubsection{Unassign a graph from a project with \t{/api/unassign-graph}}

  Like assigning a graph to a project, it can be unassigned as well using
  \t{/api/unassign-graph} using the same parameters as
  \t{/api/assign-graph}.

  The following example based on cURL shows how to perform the request:
\begin{siderules}
\begin{verbatim}
curl -X POST                                               \
     -H "Accept: application/json"                         \
     -H "Content-Type: application/json"                   \
     --cookie "SGSession=..."                              \
     --data '{ "project-uri": "http://the-project-uri",    \
               "graph-uri": "http://the-new-graph-name" }' \
     http://localhost/api/unassign-graph
\end{verbatim}
\end{siderules}

\pagebreak{}
\subsection{Running and viewing queries}

\subsubsection{Retrieve previously run queries with \t{/api/queries}}

  When running queries through SPARQLing-genomics's interface, each succesful
  query is stored in a ``query history''.  This query history can be retrieved
  using a call to \t{/api/queries}.  This resource expects a \t{GET}
  request.

  The following cURL command would retrieve the entire query history, formatted
  as XML:

\begin{siderules}
\begin{verbatim}
curl http://localhost/api/queries \
     --cookie "$COOKIE"           \
     -H "Accept: application/xml"
\end{verbatim}
\end{siderules}

\subsubsection{Toggle query marks}

\begin{sloppypar}
  Previously executed queries can be marked as \i{important}, which means
  they will survive a call to \t{/api/queries-remove-unmarked}.
\end{sloppypar}

  The following example based on cURL shows how to mark a query as
  \i{important}:
\begin{siderules}
\begin{verbatim}
curl -X POST                                               \
     -H "Accept: application/json"                         \
     -H "Content-Type: application/json"                   \
     --cookie "SGSession=..."                              \
     --data '{ "query-id": "...", "state": true }'
     http://localhost/api/query-mark
\end{verbatim}
\end{siderules}

\subsubsection{Remove unmarked queries}

  Removing unmarked queries for a project can be done by sending a
  \t{POST} request to \t{/api/queries-remove-unmarked}.
  The required parameters are:

  \begin{itemize}
    \item{\t{project-uri}: This can be obtained from a call to
      \t{/api/projects}.}
  \end{itemize}

  The following example based on cURL shows how to perform the request:
\begin{siderules}
\begin{verbatim}
curl -X POST                                               \
     -H "Accept: application/json"                         \
     -H "Content-Type: application/json"                   \
     --cookie "SGSession=..."                              \
     --data '{ "project-uri": "http://the-project-uri" }'
     http://localhost/api/queries-remove-unmarked
\end{verbatim}
\end{siderules}
