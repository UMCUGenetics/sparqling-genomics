\documentclass[11pt,a4paper]{book}
\usepackage{umcu}
\title{SPARQling structural variation}
\author{Roel Janssen}
%\date{January August 2017 --- September 2017}

\begin{document}

\begin{titlepage}
  \maketitle
  \thispagestyle{empty}
\end{titlepage}


\setcounter{page}{1}
\pagenumbering{roman}
\tableofcontents
\newpage{}

\setcounter{page}{1}
\pagenumbering{arabic}

\chapter{Introduction}

\chapter{Command-line programs}

  The project provides programs to create a complete pipeline including
  data conversion, data importing and data exploration.
  
\section{Preparing data with \texttt{vcf2turtle}}

  Obtaining variants from sequenced data is a task of so called
  \emph{variant callers}.  These programs output the variants they found in
  the \emph{Variant Call Format} (VCF).  Before we can use the data described
  in this format, we need to extract triples from it.

  The \texttt{vcf2turtle} program does exactly this, by converting a VCF file
  into the \emph{Terse RDF Triple Language} (Turtle) format.  In section
  \ref{sec:turtle2remote} (\nameref{sec:turtle2remote}) we describe how to
  import the data produced by \texttt{vcf2turtle} in the database.

\subsection{Example usage}

\begin{verbatim}
$ vcf2turtle --input-file=/path/to/my/variants.vcf > /path/to/my/variants.ttl
\end{verbatim}

\subsection{Run-time properties}

  The program typically uses less than one megabyte of memory, because it
  processes the variant calls one-by-one.  It reads through the file from top
  to bottom, so its running time is dependent on the input file size.  The
  throughput of this program is around $20$ to $60$ megabytes per second,
  depending on your computer.

\label{sec:turtle2remote}
\section{Importing data with \texttt{turtle2remote}}

  To import data in the \emph{Terse RDF Triple Language} (Turtle) format in
  a triple store (our database), we can use the \texttt{turtle2remote} program.

  The triple stores typically store data in \emph{graphs}.  One triple store
  can host multiple graphs, so we must tell the triple store which graph we
  would like to add the data to.

\subsection{Example usage}

\begin{verbatim}
$ turtle2remote --input-file=/path/to/my/variants.ttl        \
                --remote-url=http://localhost:8890           \
                --graph-iri=http://localhost:8890/MyVariants
\end{verbatim}

\chapter{Web interface}

  In addition to the command-line programs, the project provides a web
  interface for prototyping queries, and quick data reporting.  With the
  web interface you can:
  \begin{itemize}
  \item Write and execute your own SPARQL queries;
  \item Import and export data;
  \item Get a general idea about what your data looks like.
  \end{itemize}

\chapter{Programming in Python, Perl, R, Scheme, C, and C++}

\end{document}
