\chapter{Use of ontologies}
\label{chap:implemented-ontologies}

  Throughout \texttt{sparqling-genomics} we use ontologies defined by external
  working groups to increase interoperability with foreign systems.  In this
  chapter, we explain which ontologies we use, and how we use them.

\section{Dublin Core Terms}

  In the web interface, we use the Dublin Core Terms ontology \citep{dcmi-terms}
  to define organizations, collections, datasets, and samples.  Table
  \ref{table:dcterms-usage} provides an overview of the properties we use.

  \hypersetup{urlcolor=black}
  \begin{table}[H]
    \begin{tabularx}{\textwidth}{*{2}{!{\VRule[-1pt]}l}!{\VRule[-1pt]}X}
      \headrow
      \textbf{Term} & \textbf{Used as} & \textbf{Description}\\
      \evenrow
      \href{http://dublincore.org/documents/2012/06/14/dcmi-terms/\#terms-Agent}
           {dcterms:Agent}
      & Class
      & Used by the web interface to describe the organization that produced
      a collection or dataset.\\
      \oddrow
      \href{http://dublincore.org/documents/2012/06/14/dcmi-terms/\#terms-BibliographicResource}
           {dcterms:BibliographicResource}
      & Class
      & Used by the web interface to describe papers related to a dataset.\\
      \evenrow
      \href{http://dublincore.org/documents/2012/06/14/dcmi-terms/\#elements-publisher}
           {dcterms:publisher}
      & Predicate
      & Used by the web interface to identify the organization that published
      a collection or dataset.\\
      \oddrow
      \href{http://dublincore.org/documents/2012/06/14/dcmi-terms/\#terms-references}
           {dcterms:references}
      & Predicate
      & Used by the web interface to link a paper to a collection or dataset.\\
    \end{tabularx}
    \caption{\small Terms used from the Dublic Core Terms ontology.}
    \label{table:dcterms-usage}
  \end{table}
  \hypersetup{urlcolor=LinkGray}

\section{Describing genomic positions with FALDO}

  When describing the position of a nucleotide relative to its reference
  genome, we use the Feature Annotation Location Description Ontology (FALDO)
  \citep{Bolleman2016}. Table \ref{table:faldo-usage} provides an overview of
  the properties we use.

  \hypersetup{urlcolor=black}
  \begin{table}[H]
    \begin{tabularx}{\textwidth}{*{2}{!{\VRule[-1pt]}l}!{\VRule[-1pt]}X}
      \headrow
      \textbf{Term} & \textbf{Used as} & \textbf{Usage}\\
      \evenrow
      \href{http://biohackathon.org/resource/faldo\#position}{faldo:position}
      & Predicate
      & Used by \texttt{vcf2rdf} to describe the basepair position within a
      chromosome or contig.\\
      \oddrow
      \href{http://biohackathon.org/resource/faldo\#reference}{faldo:reference}
      & Predicate
      & Used by \texttt{vcf2rdf} to describe the chromosome or contig to which
      the \texttt{faldo:position} is relative to.\\
    \end{tabularx}
    \caption{\small Terms used from FALDO.}
    \label{table:faldo-usage}
  \end{table}
  \hypersetup{urlcolor=LinkGray}

\section{Custom terms}

  Sometimes we miss the right term to describe a statement.  In such cases we
  decide on a new term that is then part of the \emph{SPARQLing-genomics
    ontology}.  The use of these terms is subject to change in upcoming versions
  of \texttt{sparqling-genomics}.  Table \ref{table:sg-usage} summarizes the
  terms that are waiting to be replaced by an external ontology.

  \hypersetup{urlcolor=black}
  \begin{table}[H]
    \begin{tabularx}{\textwidth}{*{2}{!{\VRule[-1pt]}l}!{\VRule[-1pt]}X}
      \headrow
      \textbf{Term} & \textbf{Used as} & \textbf{Usage}\\
      \evenrow
      \href{http://sparqling-genomics/Origin}{sg:Origin}
      & Class
      & Used by \texttt{vcf2rdf} to point to the file or resource from which
      information was derived.\\
      \oddrow
      \href{http://sparqling-genomics/SparqlEndpoint}{sg:SparqlEndpoint}
      & Class
      & Used by the web interface to describe the endpoint that contains
      specific information.\\
      \evenrow
      \href{http://sparqling-genomics/inGraph}{sg:inGraph}
      & Class
      & Used by \texttt{vcf2rdf} to point to the file or resource from which
      information was derived.\\
      \oddrow
      \href{http://sparqling-genomics/Sample}{sg:Sample}
      & Class
      & Used by \texttt{vcf2rdf} to describe a sample.\\
      \evenrow
      \href{http://sparqling-genomics/table2rdf/Column}{table2rdf:Column}
      & Class
      & Used by \texttt{table2rdf} to describe a column.\\
      \oddrow
      \href{http://sparqling-genomics/table2rdf/Row}{table2rdf:Row}
      & Class
      & Used by \texttt{table2rdf} to describe a row.\\
      \evenrow
      \href{http://sparqling-genomics/vcf2rdf/FilterItem}{vcf2rdf:FilterItem}
      & Class
      & \\
      \oddrow
      \href{http://sparqling-genomics/vcf2rdf/HeaderItem}{vcf2rdf:HeaderItem}
      & Class
      & \\
      \evenrow
      \href{http://sparqling-genomics/vcf2rdf/InfoItem}{vcf2rdf:InfoItem}
      & Class
      & \\
      \oddrow
      \href{http://sparqling-genomics/vcf2rdf/FormatItem}{vcf2rdf:FormatItem}
      & Class
      & \\
      \evenrow
      \href{http://sparqling-genomics/vcf2rdf/VariantCall}{vcf2rdf:VariantCall}
      & Class
      & Used by \texttt{vcf2rdf} to identify a variant call.\\
      %% \oddrow
      %% \href{}{sg:}
      %% & 
      %% & \\
    \end{tabularx}
    \caption{\small Terms made up by us.}
    \label{table:sg-usage}
  \end{table}
  \hypersetup{urlcolor=LinkGray}
