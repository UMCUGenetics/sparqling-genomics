\chapter{Use of ontologies}
\label{chap:implemented-ontologies}

  In chapter \refer{chap:knowledge-graph} we described publicly accessible
  symbols as the building blocks for ontologies.  In this chapter, we explain
  which ontologies we use, and how we use them.

  The following lookup table provides the name to abbreviate an ontology,
  and the full ontology URI.

  \hypersetup{urlcolor=black}
  \begin{table}[!htbp]
    \begin{tabularx}{\textwidth}{*{1}{!{\VRule[-1pt]}l}!{\VRule[-1pt]}X}
      \headrow
      \textbf{Abbreviation} & \textbf{Ontology URI}\\
      \evenrow
      \texttt{dcterms} &
      \href{http://purl.org/dc/terms/}
           {http://purl.org/dc/terms/}\\
      \oddrow
      \texttt{dc} &
      \href{http://purl.org/dc/elements}{http://purl.org/dc/elements}\\
      \evenrow
      \texttt{faldo} &
      \href{http://biohackathon.org/resource/faldo\#}
      {http://biohackathon.org/resource/faldo\#}\\
      \oddrow
      \texttt{pato} &
      \href{http://purl.obolibrary.org/obo/}
           {http://purl.obolibrary.org/obo/}\\
      \evenrow
      \texttt{prov} &
      \href{http://www.w3.org/ns/prov\#}
           {http://www.w3.org/ns/prov\#}\\
      \oddrow
      \texttt{sg} &
      https://www.sparqling-genomics.org/\sgversion{}/\\
      \evenrow
      \texttt{bam2rdf} & \texttt{sg://\sgversion{}/bam2rdf/}\\
    \end{tabularx}
    \caption{\small Lookup table for ontology URIs and their abbreviations.}
    \label{table:ontology-abbreviations}
  \end{table}
  \hypersetup{urlcolor=LinkGray}

\section{Describing genomic positions with FALDO}

  When describing the position of a nucleotide relative to its reference
  genome, we use the Feature Annotation Location Description Ontology (FALDO)
  \citep{Bolleman2016}. Table \ref{table:faldo-usage} provides an overview of
  the properties we use.

  \hypersetup{urlcolor=black}
  \begin{table}[!htbp]
    \begin{tabularx}{\textwidth}{*{2}{!{\VRule[-1pt]}l}!{\VRule[-1pt]}X}
      \headrow
      \textbf{Term}     & \textbf{Used as} & \textbf{Usage}\\
      \evenrow
      \faldo{position}  & Predicate        & Used by \program{vcf2rdf} to
                                             describe the basepair position
                                             within a chromosome or contig.\\
      \oddrow
      \faldo{reference} & Predicate        & Used by \program{vcf2rdf} to
                                             describe the chromosome or contig
                                             to which the \faldo{position} is
                                             relative to.\\
    \end{tabularx}
    \caption{\small Terms used from FALDO.}
    \label{table:faldo-usage}
  \end{table}
  \hypersetup{urlcolor=LinkGray}

\section{Dublin Core Terms}

  In our published datasets and the \program{sg-web} interface, we use the Dublin
  Core Terms ontology \citep{dcmi-terms} to define organizations, collections,
  datasets, and samples.  Table \ref{table:dcterms-usage} provides an overview of
  the properties we use.

  \hypersetup{urlcolor=black}
  \begin{table}[!htbp]
    \begin{tabularx}{\textwidth}{*{2}{!{\VRule[-1pt]}l}!{\VRule[-1pt]}X}
      \headrow
      \textbf{Term}   & \textbf{Used as} & \textbf{Description}\\
      \evenrow
      \dcterms{Agent}       & Object     & Used by \texttt{sg-web} to describe
                                           the user or organization that
                                           produced a collection or dataset.\\
      \oddrow
      \dctype{Collection}   & Object     & Used by the portal page in the web
                                           interface as a filter mechanism to
                                           browse \dctype{Dataset}s.\\
      \evenrow
      \dctype{Dataset}      & Object     & Used by the portal page in the web
                                           interface to describe data in a
                                           graph.\\
      \oddrow
      \dcterms{date}        & Object     & Used by \program{sg-web} to describe
                                           the time a query was run.\\
      \evenrow
      \dcterms{isPartOf}    & Predicate  & Used by the portal page in the web
                                           interface to express that a
                                           \dctype{Dataset} is linked to a
                                           \dctype{Collection}.\\
      \oddrow
      \dcterms{title}       & Predicate  & Used by \program{sg-web} to name a
                                           \dctype{Collection} or a
                                           \dctype{Dataset}.\\
      \evenrow
      \dcterms{publisher}   & Predicate  & Used by \program{sg-web} to identify
                                           the organization that published
                                           a collection or dataset.\\
      \oddrow
      \dcterms{description} & Predicate  & Used by \program{sg-web} to provide
                                           a textual description of a
                                           \dctype{Collection} or a
                                           \dctype{Dataset}.\\
    \end{tabularx}
    \caption{\small Terms used from the Dublic Core Terms ontology.}
    \label{table:dcterms-usage}
  \end{table}
  \hypersetup{urlcolor=LinkGray}

\section{Provenance tracking with PROV-O}

  To maintain an understanding of ``who did what?'' when it comes to importing,
  modifying, removing, and querying data in the knowledge graph, we use parts of
  the PROV ontology \citep{prov-o}.

  \hypersetup{urlcolor=black}
  \begin{table}[!htbp]
    \begin{tabularx}{\textwidth}{*{2}{!{\VRule[-1pt]}l}!{\VRule[-1pt]}X}
      \headrow
      \textbf{Term}        & \textbf{Used as} & \textbf{Usage}\\
      \evenrow
      \prov{startedAtTime} & Predicate        & Used by \program{sg-web} to
                                                describe the date and time a
                                                query was run.\\
      \oddrow
      \prov{endedAtTime}   & Predicate        & Used by \program{sg-web} to
                                                describe the date and time a
                                                query finished.\\
    \end{tabularx}
    \caption{\small Terms used from PROV-O.}
    \label{table:prov-usage}
  \end{table}
  \hypersetup{urlcolor=LinkGray}

\section{Phenotype And Trait Ontology}

  When describing phenotype properties, we use the Phenotype And Trait
  Ontology (PATO) \citep{pato}.  Table \ref{table:pato-usage} provides an
  overview of the properties we use.

  \hypersetup{urlcolor=black}
  \begin{table}[!htbp]
    \begin{tabularx}{\textwidth}{*{2}{!{\VRule[-1pt]}l}!{\VRule[-1pt]}X}
      \headrow
      \textbf{Term}       & \textbf{Used as} & \textbf{Usage}\\
      \evenrow
      \obo{PATO\_0000384} & Object           & Used in the published datasets
                                               to describe a donor as male.\\
      \oddrow
      \obo{PATO\_0000383} & Object           & Used in the published datasets
                                               to describe a donor as female.\\
      \evenrow
      \obo{PATO\_0001894} & Predicate        & Used in the published datasets
                                               to describe the phenotypic sex.\\
    \end{tabularx}
    \caption{\small Terms used from PATO.}
    \label{table:pato-usage}
  \end{table}
  \hypersetup{urlcolor=LinkGray}

\section{Custom terms}

  Sometimes we miss the right term to describe a statement.  In such cases we
  decide on a new term that is then part of the \emph{SPARQLing-genomics
    ontology}.  The use of these terms is subject to change in upcoming versions
  of \texttt{sparqling-genomics}.  Table \ref{table:sg-usage} summarizes the
  terms that are waiting to be replaced by an external ontology.

  \hypersetup{urlcolor=black}
  \begin{table}[!htbp]
    \begin{tabularx}{\textwidth}{*{2}{!{\VRule[-1pt]}l}!{\VRule[-1pt]}X}
      \headrow
      \textbf{Term} & \textbf{Used as} & \textbf{Usage}\\
      \evenrow
      \sg{Origin}   & Object           & Used by the command-line tools (see
                                         chapter \refer{chap:command-line})
                                         to point to the file or resource from
                                         which information was derived.\\
      \oddrow
      \sg{Sample}   & Object           & Used by \program{vcf2rdf} and
                                         \program{bam2rdf} to describe a
                                         sample.\\
    \end{tabularx}
    \caption{\small Custom terms used by SPARQLing-genomics.}
    \label{table:sg-usage}
  \end{table}
  \hypersetup{urlcolor=LinkGray}
