\chapter{Information management with SPARQL}

  In chapter \ref{chap:information-retrieval} {\color{LinkGray}`%
  \nameref{chap:information-retrieval}'} we discussed how to ask questions
  to SPARQL endpoints.  In this chapter we will look at how we can modify
  the data in SPARQL endpoints.

  Using SPARQL, we can write ``layer 1'' programs --- programs that use RDF,
  and generate more RDF.

  Like the queries from chapter \ref{chap:information-retrieval} {\color{LinkGray}`%
  \nameref{chap:information-retrieval}'}, the examples can be readily used in
  the query editor of the web interface (see chapter \ref{chap:web-interface}
  {\color{LinkGray}`\nameref{chap:web-interface}'}).

\section{Managing data in graphs}

  A simple way to subset data is to put triples in separate graphs.  When
  uploading RDF data to an RDF store, we must provide a graph name, so this
  sort of works by default.

  Sometimes we'd like to remove a graph altogether to make space for new
  datasets.  For this purpose we can use the \texttt{CLEAR GRAPH} query:

\begin{siderules}
\begin{verbatim}
CLEAR GRAPH <http://example>
\end{verbatim}
\end{siderules}

  After executing this query, all triples in the graph identified by
  \texttt{<http://example>} will be sent to a pieceful place where they
  cannot be accessed anymore.

  The \texttt{CLEAR GRAPH} query is equivalent to the more elaborate:

\begin{siderules}
\begin{verbatim}
DELETE { ?s ?p ?o }
FROM <http://example>
WHERE { ?s ?p ?o }
\end{verbatim}
\end{siderules}

  Using the \texttt{DELETE} construct, we can be more specific about which
  triples to remove from a graph by filling in one of the variables.

\section{Storing inferences in new graphs}

  Calculating inferences from a large amount of data can take a lot of time.
  To avoid calculating inferences over and over again, we can store the
  inferred information as triplets.  The following example attempts to infer
  the gender related to a sample by looking at whether there's a mutation in
  the Y-chromosome.

\begin{siderules}
\begin{verbatim}
PREFIX rdf: <http://www.w3.org/1999/02/22-rdf-syntax-ns#>
PREFIX sg:  <http://sparqling-genomics/>
PREFIX faldo: <http://biohackathon.org/resource/faldo#>
PREFIX hg19:  <http://rdf.biosemantics.org/data/genomeassemblies/hg19#>

SELECT ?sample
FROM <http://hmf/variant_calling>
WHERE {
  ?sample   rdf:type        sg:Sample .
  ?variant  sg:sample       ?sample ;
            faldo:reference hg19:chrY .
}
\end{verbatim}
\end{siderules}

 Each \texttt{sample} returned by this query must've originated from a male
 donor, because it has a Y-chromosome (and also a mutation on the
 Y-chromosome).  Note that we cannot distinguish between females and males
 without a mutation on the Y-chromosome with this data, so we cannot accurately
 determine the gender for other samples.

 For the samples that definitely originated from a male donor, we can add a
 triplet in the form:

\begin{siderules}
\begin{verbatim}
   <sample-URI>  sg:gender sg:Male .
\end{verbatim}
\end{siderules}

  To do so, we use the \texttt{INSERT} construct:

\begin{siderules}
\begin{verbatim}
PREFIX rdf: <http://www.w3.org/1999/02/22-rdf-syntax-ns#>
PREFIX sg:  <http://sparqling-genomics/>
PREFIX faldo: <http://biohackathon.org/resource/faldo#>
PREFIX hg19:  <http://rdf.biosemantics.org/data/genomeassemblies/hg19#>

INSERT {
  GRAPH <http://meta> {
    ?sample   sg:gender        sg:Male .
  }
}
WHERE {
  GRAPH <http://hmf/variant_calling> {
    ?sample   rdf:type         sg:Sample .
    ?variant  sg:sample        ?sample ;
              faldo:reference  hg19:chrY .
  }
}
\end{verbatim}
\end{siderules}

  After which we can query for samples that definitely originated from a male
  donor using the following query:

\begin{siderules}
\begin{verbatim}
PREFIX rdf: <http://www.w3.org/1999/02/22-rdf-syntax-ns#>
PREFIX sg:  <http://sparqling-genomics/>
PREFIX faldo: <http://biohackathon.org/resource/faldo#>
PREFIX hg19:  <http://rdf.biosemantics.org/data/genomeassemblies/hg19#>

SELECT (COUNT (DISTINCT ?sample)) AS ?samples
FROM <http://hmf/variant_calling>
FROM <http://meta>
WHERE {
  ?sample  rdf:type   sg:Sample ;
           sg:gender  sg:Male .
}
\end{verbatim}
\end{siderules}

  We recommend keeping inferences (layer 1) in separate graphs than observed
  data (layer 0) because it allows users to turn off inferences by simply not
  including the inference graph.

%  Inferences can be drawn from inferences, which can be classified as
%  ``layer 2'' or up.  The dependency graph of individial inference programs
%  can be used as a measure for the complexity of the knowledge graph.
