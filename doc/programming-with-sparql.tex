\chapter{Using SPARQL with other programming languages}
\label{chap:programming}

\section{Using SPARQL with R}

  Before we can run a query, we need to install the \t{jsonlite} package,
  and the \t{curl} package.

\begin{siderules}
\begin{verbatim}
install.packages(c("jsonlite", "RCurl"))
\end{verbatim}
\end{siderules}

\subsection{Perform a SPARQL query using RCurl}

  Once the packages are installed, we can perform a HTTP request using RCurl
  against an endpoint managed by SPARQLing-genomics.

\begin{siderules}
\begin{verbatim}
library(RCurl)

accumulator = basicTextGatherer()
endpoint    <- "http://localhost:8001/sparql"
projectHash <- "<project-hash>"
query       <- "SELECT ?s ?p ?o WHERE { ?s ?p ?o } LIMIT 5"
cookie      <- "SGSession=<session-token>"

h$reset()
curlPerform(url           = paste0(endpoint, "?project-hash=", projectHash),
            httpheader    = c("Accept"       = "application/json",
                              "Cookie"       = cookie,
                              "Content-Type" = "application/sparql-update"),
            customrequest = "POST",
            postfields    = query,
            writefunction = accumulator$update)

jsonData    <- accumulator$value()
\end{verbatim}
\end{siderules}

\subsection{Parsing the query output}
  Now that we have gathered the query output in JSON, we are going to turn
  the JSON response into a data frame using the \t{jsonlite} package:

\begin{siderules}
\begin{verbatim}
library(jsonlite)
data <- fromJSON(jsonData)
\end{verbatim}
\end{siderules}

\section{Using SPARQL with GNU Guile}
\label{sec:sparql-with-guile}

  For Schemers using GNU Guile, the \href{https://github.com/roelj/guile-sparql}%
  {guile-sparql}\footnote{\url{https://github.com/roelj/guile-sparql}} package
  provides a SPARQL interface.

  The package provides a \t{driver} module that communicates with the
  SPARQL endpoint, a \t{lang} module to construct SPARQL queries using
  S-expressions, and a \t{util} module containing convenience functions.

  After installation, the modules can be loaded using:

\begin{siderules}
\begin{verbatim}
(use-modules (sparql driver)
             (sparql lang)
             (sparql util))
\end{verbatim}
\end{siderules}

  Using the \t{sparql-query} function, we can execute a query:

\begin{siderules}
\begin{verbatim}
(let ((endpoint       "http://localhost:8890/sparql-auth")
      (authentication "dba:secret-password")
      (query          "SELECT DISTINCT ?p WHERE { ?s ?p ?o }"))
  (display-query-results-of
    (sparql-query query
                  #:uri    endpoint
                  #:digest authentication)))
\end{verbatim}
\end{siderules}
