\chapter{Using SPARQL with other programming languages}
\label{chap:programming}

\section{Using SPARQL with R}

  Before we can run a query, we need to install the \t{jsonlite} package,
  and the \t{curl} package.

\begin{siderules}
\begin{verbatim}
install.packages(c("jsonlite", "RCurl"))
\end{verbatim}
\end{siderules}

\subsection{Perform a SPARQL query using RCurl}

  Once the packages are installed, we can perform a HTTP request using RCurl
  against an endpoint managed by SPARQLing-genomics.

\begin{siderules}
\begin{verbatim}
library(RCurl)

accumulator = basicTextGatherer()
endpoint    <- "http://localhost:8001/sparql"
projectHash <- "<project-hash>"
query       <- "SELECT ?s ?p ?o WHERE { ?s ?p ?o } LIMIT 5"
cookie      <- "SGSession=<session-token>"

h$reset()
curlPerform(url           = paste0(endpoint, "?project-hash=", projectHash),
            httpheader    = c("Accept"       = "application/json",
                              "Cookie"       = cookie,
                              "Content-Type" = "application/sparql-update"),
            customrequest = "POST",
            postfields    = query,
            writefunction = accumulator$update)

jsonData    <- accumulator$value()
\end{verbatim}
\end{siderules}

\subsection{Parsing the query output}
  Now that we have gathered the query output in JSON, we are going to turn
  the JSON response into a data frame using the \t{jsonlite} package:

\begin{siderules}
\begin{verbatim}
library(jsonlite)
data <- fromJSON(jsonData)
\end{verbatim}
\end{siderules}

\section{Using SPARQL with GNU Guile}
\label{sec:sparql-with-guile}

  For Schemers using GNU Guile the \t{(http-post)} procedure can be used
  to perform a query.  The following code snippet serves as an example.

\begin{siderules}
\begin{verbatim}
(use-modules (web client)
             (web response)
             (ice-9 receive))

(let ((endpoint "http://localhost:8001/sparql")
      (hash     "<project-hash>")
      (query    "SELECT ?s ?p ?o WHERE { ?s ?p ?o } LIMIT 5")
      (cookie   "SGSession=<session-token>"))
  (receive (header port)
           (http-post (string-append endpoint "?project-hash=" hash)
                      #:headers
                      `((accept       . ((application/s-expression)))
                        (Cookie       . ,cookie)
                        (content-type . (application/sparql-update)))
                      #:streaming? #t
                      #:body query)
           (if (= (response-code header) 200)
               (format #t "Query output:~%~:a~%" (read port))
               (format #t "The HTTP response code was ~a." (response-code header)))
           (close-port port)))
\end{verbatim}
\end{siderules}

  The SPARQLing-genomics API can respond using an S-expression that can be directly
  \t{read} in Scheme.
